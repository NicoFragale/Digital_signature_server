\section{Digital Signature Server}

\subsection{\_\_init\_\_}

Questa è la funzione costruttore della classe.
\begin{itemize}
    \item \textbf{host='0.0.0.0':} L’indirizzo IP su cui il server ascolterà le connessioni.  Il valore '0.0.0.0' indica che il server sarà accessibile su tutte le interfacce di rete disponibili del computer (locale, LAN, Internet).
    \item \textbf{port=5000:} La porta su cui il server ascolterà le connessioni in ingresso. 
    \item \textbf{certfile='server-cert.pem'}
    \item \textbf{keyfile='server-key.pem'}
\end{itemize} 

\begin{itemize}
    \item \textbf{self.server\_socket = socket.socket(...): } Crea un nuovo socket TCP/IP.
    \begin{itemize}
        \item \textbf{socket.AF\_INET: } Specifica che il server utilizzerà l’IPv4 per la comunicazione di rete.
        \item \textbf{socket.SOCK\_STREAM: }  Indica che il socket è di tipo stream (flusso)
    \end{itemize}
    \item \textbf{self.server\_socket.bind(...): } Associa il socket a una porta e un indirizzo specifici sul server, (0.0.0.0 e 5000).
    \item \textbf{self.server\_socket.listen(5): } unghezza massima della coda di connessioni in attesa.
\end{itemize}